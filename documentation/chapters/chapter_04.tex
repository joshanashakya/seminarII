\section{Dataset}
The experiment is made on Reuters-21578 \cite{reuters} dataset, in which five categories and 560 documents are used. The Reuters-21578 collection consists of 21578 documents and 135 categories. 392 documents are used as training samples and the remaining 168 documents are used as testing samples.

\begin{table}[H]
\textbf{\caption {Training set and Testing set}}
\vspace{5pt}
\begin{tabular}{|c|c|c|c|c|c|} \hline
Category & corn & cotton & rice & soybean & wheat \\ \hline
Training set & 149 & 43 & 43 & 77 & 209 \\ \hline
Testing set & 74 & 19 & 24 & 34 & 78 \\ \hline
\end{tabular}
\end{table}

\begin{table}[H]
\textbf{\caption{Dataset and Categories}}
\vspace{5pt}
\begin{tabular}{|c|c|c|c|c|c|} \hline
No. of categories & one & two & three & four & five \\ \hline
Training set & 298 & 66 & 23 & 3 & 2 \\ \hline
Testing set & 132 & 18 & 13 & 3 & 2 \\ \hline
\end{tabular}
\end{table}

\begin{table}[H]
\textbf{\caption{Sample Dataset}}
\vspace{5pt}
\begin{tabular}{|c|p{2.75in}|c|c|c|c|c|} \hline
doc\_id & text & corn & cotton & rice & soybean & wheat \\ \hline
1 & USDA REPORTS 10.572 MLN ACRES IN CONSERVATION The U.S. Agriculture Department has accepted 10,572,402 more acres of highly erodable cropland into the Conservation Reserve Program, USDA announced.     In the latest signup, farmers on 101,020 farms ... & 1 & 0 & 0 & 0 & 0 \\ \hline
13 & TRADE SEES STEADY CORN/WHEAT EXPORT INSPECTIONS The USDA's weekly export inspection report is expected to show steady corn and wheat exports and lower soybean exports, according to CBT floor traders' forecasts.     Traders ... & 1 & 0 & 0 & 1 & 1 \\ \hline
\end{tabular}
\end{table}

\section{Implementation Environment}
Anaconda distribution of Python 3.6 is used as programming language. The other libraries used are:
\begin{enumerate}[i.]
\item numpy: To manage list of data
\item pandas: To manipulate data
\item nltk: To preprocess data
\item scikit-learn: To split dataset into training and testing set, to perform TF-IDF vectorization, and to compute hamming loss
\item pystruct: To perform multi-label classification using structured SVM
\end{enumerate}

\section{Parameters}
\begin{enumerate}[i.]
\item The vocabulary size is 3827.
\item The inference method method used is qpbo.
\item The inference\_cache is 50 as in pystruct documentation. If  $>$ 0 the most violating of the cached examples will be used to construct a global constraint.
\item The cost of constraints violation, $C$ is taken $0.1$.
\item The convergence tolerance, tol is taken 0.01 as in pystruct documentation. If dual objective decreases less than tol, learning is stopped. The default corresponds to ignoring the behavior of the dual objective and stop only if no more constraints can be found.
\item The maximum number of iterations defined in pystruct is 10000.
\end{enumerate}

\section{Example}
A news entitled “China maintains strong demand for U.S. corn, soy: USDA” is taken as sample document from Reuters website\cite{reuters-ex} to perform structured SVM multi-label classification. The sample is:

CHICAGO (Reuters) - Chinese demand for U.S. corn and soybeans remained robust in the latest week, U.S. Agriculture Department (USDA) data showed, and traders expect the recent surge of deals will cause the U.S. government to boost its export forecast for both commodities.
\\\\
A USDA report released Friday showed that export sales of soybeans to China totaled 1.608 million tonnes in the week ended Sept. 3, the latest reporting period. Weekly corn export sales to China were 1.137 million tonnes. [EXP/SOY] [EXP/COR]
\\\\
Separately, the government said private exporters reported the sale of 262,000 tonnes of soybeans to China, the sixth straight trading day that an announcement of a sale to the world’s top buyer of the oilseed has been announced.
\\\\
China also has been ramping up its U.S. corn imports, as the country faces its first real corn shortfall of corn in years. A sharp price surge in corn - critical for China’s mammoth hog, dairy and poultry sectors - is the latest in a series of ructions that include a devastating pig disease, pandemic-driven upsets for international suppliers and warnings of a growing food supply gap.
\\\\
“The trade is pricing in expectations that USDA will cut the size of this year’s corn and soybean crops while increasing demand estimates due to the recent Chinese buying spree,” Arlan Suderman, chief commodities economist for broker StoneX, said in note to clients.
\\\\
USDA will give an update on the export outlook in its monthly World Agricultural Supply and Demand Estimates report at 11 a.m. CDT (1600 GMT) on Friday.
\\\\
China vowed to import \$36.5 billion of U.S. agricultural goods annually as part of a Phase 1 trade deal signed in January. Chinese purchases through the first seven months of the year totaled just \$8.559 billion, according to U.S. Census Bureau trade data.
\\\\
The outputs of 3 steps in structured SVM multi-label classification are as follows:

Step 1: Preprocessing (tokenize, stopword removal, porter stemming)

chicago reuter chines demand us corn soybean remain robust latest week us agricultur depart usda data show trader expect recent surg deal caus us govern boost export forecast commoditiesa usda report releas friday show export sale soybean china total million tonn week end sept latest report period weekli corn export sale china million tonn expsoy expcorsepar govern said privat export report sale tonn soybean china sixth straight trade day announc sale world top buyer oilse announcedchina also ramp us corn import countri face first real corn shortfal corn year sharp price surg corn critic china mammoth hog dairi poultri sector latest seri ruction includ devast pig diseas pandemicdriven upset intern supplier warn grow food suppli gapth trade price expect usda cut size year corn soybean crop increas demand estim due recent chines buy spree arlan suderman chief commod economist broker stonex said note clientsusda give updat export outlook monthli world agricultur suppli demand estim report cdt gmt fridaychina vow import billion us agricultur good annual part phase trade deal sign januari chines purchas first seven month year total billion accord us censu bureau trade data
\\\\
Step 2: Vectorization

Non-zero TF-IDF values of the words are:

0.0517, 0.0757, 0.0414, 0.0510, 0.0594, 0.1003, 0.1101, 0.0717, 0.0939, 0.0763, 0.0440, 0.0717, 0.0560, 0.1003, 0.0661, 0.1003, 0.2115, 0.2113, 0.0450, 0.2532, 0.0429, 0.0731, 0.0360, 0.0560, 0.0828, 0.1564, 0.0606, 0.1385, 0.1763, 0.0329, 0.0893, 0.0803, 0.0489, 0.0893, 0.0460, 0.0871, 0.0844, 0.1418, 0.0717, 0.1066, 0.0565, 0.0546, 0.0782, 0.0692, 0.0893, 0.0546, 0.0880, 0.0620, 0.1003, 0.0767, 0.0463, 0.0420, 0.0613, 0.0628, 0.1984, 0.2007, 0.0420, 0.0782, 0.0582, 0.0652, 0.0746, 0.0537, 0.0560, 0.1003, 0.1003, 0.0939, 0.0696, 0.0542, 0.0455, 0.0893, 0.1026, 0.0661, 0.0565, 0.1464, 0.0161, 0.0365, 0.1723, 0.0746, 0.0939, 0.0893, 0.0576, 0.0782, 0.0939, 0.1241, 0.0704, 0.0939, 0.0857, 0.1588, 0.1003, 0.0985, 0.0717, 0.1877, 0.0764, 0.0803, 0.0801, 0.1489, 0.0571, 0.1003, 0.1435, 0.1072, 0.0828, 0.0798, 0.0652, 0.0972, 0.0904
\\\\
Step 3: Classification

The classification output of the sample document is 'corn'.
